\documentclass[12pt, letter paper]{article}
\usepackage[utf8]{inputenc}
\usepackage{amsmath}
\usepackage{geometry}
\usepackage{fancyhdr}
\usepackage{graphicx}
\geometry{
left = 1in,
right = 1in,
top = 1in,
bottom = 1in
}
\title{Homework 8 Solutions}
\author{Chris Logan, Ben Lanoue}
\date{2/12/19}

\begin{document}
\maketitle
$\mathbf{Section  2.4}$
\begin{enumerate}
	\setcounter{enumi}{1}
	\item 
    	\begin{enumerate}
		\item 128
		\item 2
		\item 7
		\item -128
	\end{enumerate}
	
	\setcounter{enumi}{3}
	\item
	\begin{enumerate}
		\item 1, -2, 4, 8
		\item 3, 3, 3, 3
		\item 8, 11, 23, 71
		\item 2, 0, 8, 0
	\end{enumerate}
	
	\setcounter{enumi}{13}
	\item
	\begin{enumerate}
		\item \( a_0 = 3; a_n = a_{n-1} \)
		\item \(a_0 = 0; a_n = a_{n-1} +2\)
		\item \(a_0 = 3; a_n = a_{n-1} +2\)
		\item \(a_0 = 1; a_n = a_{n-1} *  5\)
		\item \(a_0 = 0; a_n = a_{n-1} + 2n - 1\)
		\item \(a_0 = 0; a_n = a_{n-1} - +2n\) 
		\addtocounter{enumii}{1}
		\item \(a_0 = 1; a_n = a_{a-1} * n\)
	\end{enumerate}
	
	\setcounter{enumi}{29}
	\item
	\begin{enumerate}
		\item 16
		\item 84
		\item\( \frac{176}{105} \)
		\item 4
	\end{enumerate}
	
	\setcounter{enumi}{31}
	\item
	\begin{enumerate}
		\item 10
		\item 9,330
		\item 21,215
		\item 511
	\end{enumerate}
	
\end{enumerate}

$\mathbf{Section 4.1}$
\begin{enumerate}
	\setcounter{enumi}{5}
	\item Show that if a, b, c, and d are integers, where a does not equal 0, then if  a | c and\\ b | d, then ab | cd
	\\
	\\We know that for an integer to be divisible, there must exist an integer such that x = yz so we can say y 		divides x.
	\\
	\\First we know that a | c, so there must exist an integer f such that
	\\c = af
	\\
	\\Next we know that b | d, so there must exist an integer g such that
	\\d = bg
	\\
	\\Now we can say that 
	\\cd = af * bd
	\\Which we can now rearrange to
	\\cd = ab * fg
	\\
	\\Since we know f is and integer and g is an integer, fg must also be an integer (lets call it h). So that means 	were left with
	\\cd = ab * h
	\\Which according to the definition of divides, means ab | cd
	
	\setcounter{enumi}{23}
	\item
	\begin{enumerate}
		\item -3
		\item -12
		\item 94
	\end{enumerate}
	
	\item
	\begin{enumerate}
		\setcounter{enumii}{1}
		\item -7
		\item 140
	\end{enumerate}
	
	\item 16, 28, 40, 52, 64
	
	\setcounter{enumi}{27}
	\item
	\begin{enumerate}
		\item no
		\item yes
		\item no
		\item yes
	\end{enumerate}
	
	\item
	\begin{enumerate}
		\item no
		\item no
		\item yes
		\item no
	\end{enumerate}

\end{enumerate}

$\mathbf{Section  4.2}$

\begin{enumerate}
	\setcounter{enumi}{4}
	\item
	\begin{enumerate}
		\item 1 0111 1010
		\item 11 1000 0100
	\end{enumerate}
	
	\item
	\begin{enumerate}
		\item 1111 0111
		\item 5252
	\end{enumerate}
	
	\item
	\begin{enumerate}
		\item 1000 0000 1100
		\item 1 0011 0101 1010 1011
	\end{enumerate}
	
	\item 1011 1010 1101 1111 1010 1100 1110 1101
	\\
	\item 1010 1011 1100 1101 1110 1111
	\\
	\item
	\begin{enumerate}
		\item F7
		\item AAA
	\end{enumerate}
	
	\item B7B
	\\
	\item 1863
	\\
	\setcounter{enumii}{17}
	
	\item First convert each hexadecimal place into its corresponding binary value. Each heaxadecimal place should represent 4 binary digits
	\\
	\\ Next, after converting, from right to left, group each binary digit into groups of 3
	\\
	\\ Finally, convert each group of 3 binary digits into its corresponding octal counterpart to get the final conversion from hexadecimal to octal
	\\
	\\ Example
	\\ Converting \(AB_{16}\) into octal
	\\ B = 1011
	\\A = 1010
	\\AB = 1010 1011
	\\
	\\Now regroup it
	\\  10 101 011
	\\
	\\Now reconvert the corresponding binary digits to octal (or decimal, the results will still be the same)
	\\ 253
	
	
	
\end{enumerate}
	

				

\end{document}